\documentclass{article}
\usepackage[utf8]{inputenc}
\usepackage[brazil]{babel}
\usepackage{epsfig}
\usepackage{fancyhdr}
\usepackage{indentfirst} %In­dent first para­graph af­ter sec­tion header
\usepackage{titlesec}
\usepackage{amsmath}
\usepackage{amsthm}

\pagestyle{empty}

\headheight 40mm      %
\oddsidemargin 2.0mm  %
\evensidemargin 2.0mm %
\topmargin -40mm      %
\textheight 250mm     %
\textwidth 160mm      %
%
\newcounter{execs}
\setcounter{execs}{0}
\newcommand{\exec}[0]{\addtocounter{execs}{1}\item[\textbf{\arabic{execs}.}]}

\fancypagestyle{first}
{
\pagestyle{fancy}
}
%%%%%%%%%%%%%%%%%%%%%%%%%%%%%%%%%%%%%%%%%%%%%%%%%%%%%%%%
%%%%%%%%%%%%%%%%%%%%%%%%%%%%%%%%%%%%%%%%%%%%%%%%%%%%%%%%
% PLEASE, EDIT THIS!
\fancyhead[LO]{\small $4^a$ Lista \\ 
                DCC008 - Cálculo Numérico  \\
                \textbf{Entrega: 28 de Outubro de 2018} }

\fancyhead[RO]{\small Universidade Federal de Juiz de Fora - UFJF \\ 
                Departamento de Ciência da Computação \\
               \textit{Nome: Aluno 1}\\
               \textit{Nome: Aluno 2}}


\begin{document}
\thispagestyle{first}
\noindent \textbf{Obs1.:}  Escolha um ou mais métodos de interpolação dado em aula para resolver os problemas abaixo.

\noindent \textbf{Obs2.:}  Discuta os resultados.

\begin{itemize}

\exec Considere a função
$$
f(x)=\dfrac{1}{1+25x^2}
$$
definida no intervalo $x\in [a,b]$ com $a=-1$ e $b=1$. 

\begin{itemize}

\item[a)] Seja $P_n(x)$ o polinômio que interpola $f(x)$ nos pontos
$$
x_k = a + \dfrac{b-a}{n}k, \quad k = 0,1,2,...,n
$$
igualmente espaçados no intervalo $[a,b]$. Nesse contexto,
apresente gráficos comparando $P_n(x)$ com $f(x)$ para $n= 2,3,4,5,6,7,8,9,10$.

\item [b)] Repita o item (a), porém utilizando os pontos
$$
x_k = \dfrac{a+b}{2} - \dfrac{b-a}{2}\cos\left(\dfrac{k}{n}\pi \right), \quad k = 0,1,2,...,n.
$$
igualmente espaçados no intervalo $[a,b]$.
\item [c)] Repita o item (a), considerando  uma interpolação \textbf{linear por partes} com nós
$$
x_k = -1 + \dfrac{2}{n}k, \quad k = 0,1,2,...,n.
$$
igualmente espaçados no intervalo $[a,b]$.

\item [d)] Calcule o erro de interpolação na norma do máximo $\|f(x)-P_n(x)\|_{\infty} $ e construa uma tabela comparando os resultados obtidos nos itens (a), (b) e (c) para $n = 2,5,10$. Comente os resultados.
%$$
%\|f(x)-P_n(x)\| = \sqrt{\int_a^b |f(x)-P_n(x)|^2}
%$$
%\textbf{obs.:} Para o caso da interpolação linear por partes, calcule o erro de cada intervalo, separadamente, e some os resultados.

\end{itemize}

\end{itemize}

\end{document}
