\documentclass{article}
\usepackage[utf8]{inputenc}
\usepackage[brazil]{babel}
\usepackage{epsfig}
\usepackage{fancyhdr}
\usepackage{indentfirst} %In­dent first para­graph af­ter sec­tion header
\usepackage{titlesec}
\usepackage{amsmath}
\usepackage{amsthm}

\pagestyle{empty}

\headheight 40mm      %
\oddsidemargin 2.0mm  %
\evensidemargin 2.0mm %
\topmargin -40mm      %
\textheight 250mm     %
\textwidth 160mm      %
%
\newcounter{execs}
\setcounter{execs}{0}
\newcommand{\exec}[0]{\addtocounter{execs}{1}\item[\textbf{\arabic{execs}.}]}

\fancypagestyle{first}
{
\pagestyle{fancy}
}
%%%%%%%%%%%%%%%%%%%%%%%%%%%%%%%%%%%%%%%%%%%%%%%%%%%%%%%%
%%%%%%%%%%%%%%%%%%%%%%%%%%%%%%%%%%%%%%%%%%%%%%%%%%%%%%%%
% PLEASE, EDIT THIS!
\fancyhead[LO]{\small $5^a$ Lista \\ 
                DCC008 - Cálculo Numérico  \\
                \textbf{Entrega: 11 de Novembro de 2018} }

\fancyhead[RO]{\small Universidade Federal de Juiz de Fora - UFJF \\ 
                Departamento de Ciência da Computação \\
               \textit{Nome: Aluno 1}\\
               \textit{Nome: Aluno 2}}


\begin{document}
\thispagestyle{first}
%    \noindent \textbf{Obs1.:}  Escolha um ou mais métodos de interpolação dado em aula para resolver os problemas abaixo.
%    
%    \noindent \textbf{Obs2.:}  Discuta os resultados.

\begin{itemize}

\exec O arquivo ``dados.txt'' contém os dados históricos referentes a cotação diária das ações da empresa Petrobras (PETR4) nos últimos 2 anos, que são negociadas na bolsa de valores de São Paulo (BOVESPA). 

\begin{itemize}

\item[a)] Apresente gráficos comparando os dados do arquivo ``dados.txt'' com as curvas ajustadas pelo método de mínimos quadrados para diferentes ordens polinomiais ($P_n(x)$, $n=1,3,5,10,15,20,50,100$).

\item[b)] Definindo como $x$ a primeira coluna e $y$ a segunda coluna do arquivo ``dados.txt'', calcule, para todos as ordens polinomiais do item (a), o coeficiente de determinação $r$ que pode ser calculado como:
$$
r^2= 1 - \dfrac{\displaystyle \sum_{i=1}^{k} \left(y_i-P_n(x_i) \right)^2}{\displaystyle \sum_{i=1}^{k} y_i^2 - \dfrac{1}{k} \left(\sum_{i=1}^{k} y_i \right)^2 }
$$
onde $k$ denota a quantidade de dados do arquivo ``dados.txt''. Monte uma tabela apresentando os resultados do coeficiente de determinação $r$ em porcentagem ($r*100$).

\item[c)] A partir dos resultados da letra (b), utilize a curva que melhor se adapte aos dados fornecidos para projetar os preços da ação para os próximos 100 dias e apresente um gráfico com este resultado. 

\end{itemize}

\end{itemize}

\end{document}
