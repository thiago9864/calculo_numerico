\documentclass{article}
\usepackage[utf8]{inputenc}
\usepackage[brazil]{babel}
\usepackage{epsfig}
\usepackage{fancyhdr}
\usepackage{indentfirst} %In­dent first para­graph af­ter sec­tion header
\usepackage{titlesec}
\usepackage{amsmath}
\usepackage{amsthm}

\pagestyle{empty}

\headheight 40mm      %
\oddsidemargin 2.0mm  %
\evensidemargin 2.0mm %
\topmargin -40mm      %
\textheight 250mm     %
\textwidth 160mm      %
%
\newcounter{execs}
\setcounter{execs}{0}
\newcommand{\exec}[0]{\addtocounter{execs}{1}\item[\textbf{\arabic{execs}.}]}

\fancypagestyle{first}
{
\pagestyle{fancy}
}
%%%%%%%%%%%%%%%%%%%%%%%%%%%%%%%%%%%%%%%%%%%%%%%%%%%%%%%%
%%%%%%%%%%%%%%%%%%%%%%%%%%%%%%%%%%%%%%%%%%%%%%%%%%%%%%%%
% PLEASE, EDIT THIS!
\fancyhead[LO]{\small $5^a$ Lista \\ 
                DCC008 - Cálculo Numérico  \\
                \textbf{Entrega: 11 de Novembro de 2018} }

\fancyhead[RO]{\small Universidade Federal de Juiz de Fora - UFJF \\ 
                Departamento de Ciência da Computação \\
               \textit{Nome: Thiago de Almeida}\\
               \textit{Nome: Renan Nunes}}


\begin{document}
\thispagestyle{first}
%    \noindent \textbf{Obs1.:}  Escolha um ou mais métodos de interpolação dado em aula para resolver os problemas abaixo.
%    
%    \noindent \textbf{Obs2.:}  Discuta os resultados.

\begin{itemize}

\exec O arquivo ``dados.txt'' contém os dados históricos referentes a cotação diária das ações da empresa Petrobras (PETR4) nos últimos 2 anos, que são negociadas na bolsa de valores de São Paulo (BOVESPA). 

\begin{itemize}

\item[a)] Apresente gráficos comparando os dados do arquivo ``dados.txt'' com as curvas ajustadas pelo método de mínimos quadrados para diferentes ordens polinomiais ($P_n(x)$, $n=1,3,5,10,15,20,50,100$).

\item[b)] Definindo como $x$ a primeira coluna e $y$ a segunda coluna do arquivo ``dados.txt'', calcule, para todos as ordens polinomiais do item (a), o coeficiente de determinação $r$ que pode ser calculado como:
$$
r^2= 1 - \dfrac{\displaystyle \sum_{i=1}^{k} \left(y_i-P_n(x_i) \right)^2}{\displaystyle \sum_{i=1}^{k} y_i^2 - \dfrac{1}{k} \left(\sum_{i=1}^{k} y_i \right)^2 }
$$
onde $k$ denota a quantidade de dados do arquivo ``dados.txt''. Monte uma tabela apresentando os resultados do coeficiente de determinação $r$ em porcentagem ($r*100$).

\item[c)] A partir dos resultados da letra (b), utilize a curva que melhor se adapte aos dados fornecidos para projetar os preços da ação para os próximos 100 dias e apresente um gráfico com este resultado. 

\end{itemize}

\end{itemize}

\newpage

O arquivo de dados sobre a cotação diária das ações da empresa Petrobrás (PETR4) nos últimos 2 anos foi analisado e ajustado pelo método dos mínimos quadrados.

\item - Analise utilizando-se do método de Gauss.
\begin{figure}[!htb]
\includegraphics [width=5cm,height=5cm]{Gauss/G1.png}
\includegraphics [width=5cm,height=5cm]{Gauss/G3.png}
\includegraphics [width=5cm,height=5cm]{Gauss/G5.png}
\includegraphics [width=5cm,height=5cm]{Gauss/G10.png}
\includegraphics [width=5cm,height=5cm]{Gauss/G15.png}
\includegraphics [width=5cm,height=5cm]{Gauss/G20.png}
\includegraphics [width=5cm,height=5cm]{Gauss/G50.png}
\includegraphics [width=5cm,height=5cm]{Gauss/G100.png}
\end{figure}

\text No Método de Gauss, com $N = 15$, pode-se notar um grande erro do método. Formando um gráfico com um imenso erro.  O motivo desse erro é;

\newpage

\item - Analise utilizando-se do método LU.
\begin{figure}[!htb]
\includegraphics [width=5cm,height=5cm]{LU/G1.png}
\includegraphics [width=5cm,height=5cm]{LU/G3.png}
\includegraphics [width=5cm,height=5cm]{LU/G5.png}
\includegraphics [width=5cm,height=5cm]{LU/G10.png}
\includegraphics [width=5cm,height=5cm]{LU/G15.png}
\includegraphics [width=5cm,height=5cm]{LU/G20.png}
\includegraphics [width=5cm,height=5cm]{LU/G50.png}
\includegraphics [width=5cm,height=5cm]{LU/G100.png}
\end{figure}

\newpage

\text Definindo as colunas x e y para o método polinomial do item (a), o coeficiente r que pode ser calculado e expresso em porcetagem é:
\begin{table}[h]
\centering
  \begin{tabular}{l|l|lll}
    $Ordem$ $n$ & $Coeficiente$ $r$ & $Coeficiente$ $r$ $em$ $porcetagem$ $(\%)$\\
    \hline
    1 &  0.71092847081604467315 & 71.092\\
    
    3 & 0.7910095119409314065 & 79.100\\
    
    5 & 0.8750227594557078267  & 87.502 \\
    
    10 & 0.9132363941170467756 & 91.323\\
    
    15 &  \sqrt{-110898660834074.66367}    & Não Possui uma raiz real\\
    
    20 & 0.9436191040476386921 & 94.361\\
    
    50 & 0.8960730081149934023 & 89.607\\
    
    100 & 0.8433231181608608121 & 84.332 \\
    \hline
  \end{tabular}
  \caption{coeficiente r em porcetagem para o método de Gauss}
\end{table}

\begin{table}[h]
\centering
  \begin{tabular}{l|l|lll}
    $Ordem$ $n$ & $Coeficiente$ $r$ & $Coeficiente$ $r$ $em$ $porcetagem$ $(\%)$\\
    \hline
    1 &  0.71092847081604467743 & 71.092\\
    
    3 & 0.7910095119409314058 & 79.100\\
    
    5 & 0.87502275945570782376  & 87.502 \\
    
    10 & 0.91323639411704699335 & 91.323\\
    
    15 & 0.94150258009193910986    & 94.150\\
    
    20 & 0.94373292342024341523 & 94.373\\
    
    50 & 0.9656205483094383549 & 96.562\\
    
    100 & 0.88523656133428724256 & 88.523 \\
    \hline
  \end{tabular}
  \caption{coeficiente r em porcetagem para o método LU}
\end{table}

\text Todos os testes acima foram realizados em computador Mac mini (Late 2014).
Com as seguintes configurações;
\item Processador: 1,4 GHz Intel Core i5
\item Memoria: 8 GB 1600 MHz DDR3
\item Sistema Operacional: OSX 10.14.1

\newpage 
Analisando a partir dos resultados obtidos nos itens anteriores, as tentativas de previsão para os próximos 100 dias resultam em:

\item Pelo Método de Gauss:
\begin{figure}[!htb]
\includegraphics [width=5cm,height=5cm]{PrevisaoG/P1.png}
\includegraphics [width=5cm,height=5cm]{PrevisaoG/P3.png}
\includegraphics [width=5cm,height=5cm]{PrevisaoG/P5.png}
\includegraphics [width=5cm,height=5cm]{PrevisaoG/P10.png}
\includegraphics [width=5cm,height=5cm]{PrevisaoG/P15.png}
\includegraphics [width=5cm,height=5cm]{PrevisaoG/P20.png}
\includegraphics [width=5cm,height=5cm]{PrevisaoG/P50.png}
\includegraphics [width=5cm,height=5cm]{PrevisaoG/P100.png}
\end{figure}

\item Pelo Método LU:
\begin{figure}[!htb]
\includegraphics [width=5cm,height=5cm]{PrevisaoLU/P1.png}
\includegraphics [width=5cm,height=5cm]{PrevisaoLU/P3.png}
\includegraphics [width=5cm,height=5cm]{PrevisaoLU/P5.png}
\includegraphics [width=5cm,height=5cm]{PrevisaoLU/P10.png}
\includegraphics [width=5cm,height=5cm]{PrevisaoLU/P15.png}
\includegraphics [width=5cm,height=5cm]{PrevisaoLU/P20.png}
\includegraphics [width=5cm,height=5cm]{PrevisaoLU/P50.png}
\includegraphics [width=5cm,height=5cm]{PrevisaoLU/P100.png}
\end{figure}

\newpage

\text Como os gráficos demonstram, as tentativas de previsão por ambos os métodos, Gauss e LU, não funcionam bem, conseguindo uma projeção apenas para $N = 1$. Essa falha na previsão se origina do fato do método dos mínimos quadrados só funcionar bem para dados que possuam dependência linear.

\end{document}
