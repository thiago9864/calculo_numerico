\documentclass{article}
\usepackage[utf8]{inputenc}
\usepackage[brazil]{babel}
\usepackage{epsfig}
\usepackage{fancyhdr}
\usepackage{indentfirst} %In­dent first para­graph af­ter sec­tion header
\usepackage{titlesec}
\usepackage{amsmath}
\usepackage{amsthm}

\pagestyle{empty}

\headheight 40mm      %
\oddsidemargin 2.0mm  %
\evensidemargin 2.0mm %
\topmargin -40mm      %
\textheight 250mm     %
\textwidth 160mm      %
%
\newcounter{execs}
\setcounter{execs}{0}
\newcommand{\exec}[0]{\addtocounter{execs}{1}\item[\textbf{\arabic{execs}.}]}

\fancypagestyle{first}
{
\pagestyle{fancy}
}
%%%%%%%%%%%%%%%%%%%%%%%%%%%%%%%%%%%%%%%%%%%%%%%%%%%%%%%%
%%%%%%%%%%%%%%%%%%%%%%%%%%%%%%%%%%%%%%%%%%%%%%%%%%%%%%%%
% PLEASE, EDIT THIS!
\fancyhead[LO]{\small $6^a$ Lista \\ 
                DCC008 - Cálculo Numérico  \\
                \textbf{Entrega: 25 de Novembro de 2018} }

\fancyhead[RO]{\small Universidade Federal de Juiz de Fora - UFJF \\ 
                Departamento de Ciência da Computação \\
               \textit{Nome: Aluno 1}\\
               \textit{Nome: Aluno 2}}


\begin{document}
\thispagestyle{first}
%    \noindent \textbf{Obs1.:}  Escolha um ou mais métodos de interpolação dado em aula para resolver os problemas abaixo.
%    
%    \noindent \textbf{Obs2.:}  Discuta os resultados.

\begin{itemize}

\exec Seja a função:
\begin{equation}\label{solexat1}
u(x) = c_1 e^{-\tfrac{x}{\sqrt{\varepsilon}}} + c_2 e^{\tfrac{x}{\sqrt{\varepsilon}}}+1
\end{equation}
com $c_1 = -1-c_2$ e $c_2 = \dfrac{e^{-\tfrac{1}{\sqrt{\varepsilon}}}-1}
{e^{\tfrac{1}{\sqrt{\varepsilon}}}-e^{-\tfrac{1}{\sqrt{\varepsilon}}}}$.
Adotando $x\in [0,1]$ e $\varepsilon=0.001$, faça:

\begin{itemize}

\item[a)] Ajuste a função $u(x)$ por mínimos quadrados utilizandpo diferentes ordens ($n=1,2,3,4,5$) e apresente gráficos comparando com a solução exata \eqref{solexat1}. Para isso, utilize a quadratura de Gauss para integração numérica utilizando $n+1$ pontos. 

\item [b)] Repita o item (a) dividindo o intervalo $[0,1]$ em 25 subintervalos iguais e utilize ordem $n=1$ em cada subintervalo. 

\item [c)] Calcule o erro de todas as curvas ajustadas nos itens (a) e (b) utilizando a norma L2, que pode ser definida como:
$$
\|u(x)-u_h(x)\|_0 =\sqrt{ \sum_{i=0}^{K-1} \int_{x_i}^{x_{i+1}} (u(x)-u_h(x))^2~dx}
$$
onde $u_h(x)$ é a curva ajustada por mínimos quadrados e $K$ é o número de partições do domínio (neste caso, $x_0=0$ e $x_K=1$). Aplique a regra $3/8$ de simpson para o cálculo da integral. Construa uma tabela com os resultados.

\end{itemize}





\end{itemize}

\end{document}
