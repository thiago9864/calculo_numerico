\documentclass{article}
\usepackage[utf8]{inputenc}
\usepackage[brazil]{babel}
\usepackage{epsfig}
\usepackage{fancyhdr}
\usepackage{indentfirst} %In­dent first para­graph af­ter sec­tion header
\usepackage{titlesec}
\usepackage{amsmath,amssymb}
\usepackage{amsthm}
\usepackage{listings}
\usepackage{color}
\UseRawInputEncoding


\definecolor{dkgreen}{rgb}{0,0.6,0}
\definecolor{gray}{rgb}{0.5,0.5,0.5}
\definecolor{mauve}{rgb}{0.58,0,0.82}

\lstset{
  language=Python,                
  basicstyle=\footnotesize,           
  numbers=left,                   
  numberstyle=\tiny\color{gray},  
  stepnumber=2,                             
  numbersep=5pt,                  
  backgroundcolor=\color{white},    
  showspaces=false,               
  showstringspaces=false,         
  showtabs=false,                 
  frame=single,                   
  rulecolor=\color{black},        
  tabsize=2,                      
  captionpos=b,                   
  breaklines=true,                
  breakatwhitespace=false,        
  title=\lstname,                               
  keywordstyle=\color{blue},          
  commentstyle=\color{dkgreen},       
  stringstyle=\color{mauve},     
}

\pagestyle{empty}

\headheight 40mm      %
\oddsidemargin 2.0mm  %
\evensidemargin 2.0mm %
\topmargin -40mm      %
\textheight 250mm     %
\textwidth 160mm      %
%
\newcounter{execs}
\setcounter{execs}{0}
\newcommand{\exec}[0]{\addtocounter{execs}{1}\item[\textbf{\arabic{execs}.}]}

\fancypagestyle{first}
{
\pagestyle{fancy}
}


\fancyhead[LO]{\small $1^a$ Lista \\ 
                DCC008 - Cálculo Numérico  \\
                \textbf{Entrega: 27 de Agosto de 2018}\\
                \textbf{RESPOSTAS}}

\fancyhead[RO]{\small Universidade Federal de Juiz de Fora - UFJF \\ 
                Departamento de Ciência da Computação \\
               \textit{Nome: Thiago de Almeida      }\\
               \textit{Nome: Renan N C Gonçalves}}



\usepackage{graphicx}
\begin{document}
\thispagestyle{first}

\begin{itemize}

\exec Seja o problema de valor incial associado a lei de resfriamento de Newton:
\begin{eqnarray} \label{N}
\begin{cases}
 \dfrac{d \theta}{d t} = -K(\theta - \theta_m)\\ 
\theta(0) = \theta_0
\end{cases}, 
\quad t\in [0,T]
\end{eqnarray}

A solução exata para o problema \eqref{N} é dada por:
\begin{equation}\label{solexatN}
\theta(t) = (\theta_0 - \theta_m) e^{-Kt} + \theta_m
\end{equation}

\begin{itemize}
 \item[(a)] Apresente as seguintes discretizações por diferenças finitas para aproximar o problema \eqref{N}:
\begin{itemize}
\item Euler Explícito;
\item Euler Implícito;
\item Diferença Central;
\item Crank-Nicolson.
\end{itemize} 
 \item[(b)] Tomando $K=0,035871952$ $min^{-1}$, $\theta_0=99$ $^oC$ e $\theta_m=27$ $^oC$ construa gráficos ou tabelas comparando a solução exata \eqref{solexatN} com a aproximada obtida pelas discretizações do item (a) no intervalo $[0,50]$.
 
\textbf{Obs.:} Comente a escolha do $\Delta t$ para cada método de acordo com a restrição de condicionalidade.
 
 \item[(c)] A partir da escolha de um único $\Delta t$ que cumpra ao mesmo tempo a condição de estabilidade dos 4 métodos, calcule o erro entre a solução exata e a aproximada na norma do máximo $\|\theta(t_n) - \theta^n\|_{\infty}$ e comente os resultados.
 
 \item[(d)] Os resultados dos itens (b) e (c) podem variar adotando precisão simples ou dupla na declaração das variáveis? justifique apresentando resultados. 
\end{itemize} 


\exec Seja o problema de valor de contorno:
\begin{align} \label{prob1}
- \varepsilon\dfrac{d^2 u}{d x^2} + u &= 1, \quad x\in \Omega= [0,1]
 \\ \label{cc1}
 u(0)&=u(1)=0.
\end{align}
A solução exata para o problema \eqref{prob1}-\eqref{cc1} é
\begin{equation}\label{solexat1}
u(x) = c_1 e^{-\tfrac{x}{\sqrt{\varepsilon}}} + c_2 e^{\tfrac{x}{\sqrt{\varepsilon}}}+1
\end{equation}
onde $c_1 = -1-c_2$ e $c_2 = \dfrac{e^{-\tfrac{1}{\sqrt{\varepsilon}}}-1}
{e^{\tfrac{1}{\sqrt{\varepsilon}}}-e^{-\tfrac{1}{\sqrt{\varepsilon}}}}$.

\begin{itemize}

\item[a)] Apresente uma discretização pelo método de diferenças finitas de segunda ordem implícito para o problema \eqref{prob1}. Implemente o método desenvolvido e utilize o algoritmo de Thomas para a resolução da matriz.
  
  
\item[b)] compare a solução exata e a aproximada para $\varepsilon = 0.1, 0.01, 0.001, 0.0001$
para um único valor de $h$.

\item[c)] apresente um gráfico da taxa de convergência comparando o erro 
entre a solução exata e a aproximada para malhas formadas por $4^i$ elementos, 
com $i=1,2,3,4,5$, utilizando a norma do máximo $\|u(x_n) - u^n\|_{\infty}$ para $\varepsilon = 0.1, 0.01, 0.001, 0.0001$ (apresente todos os resultados no mesmo gráfico).

 \item[(d)] Os resultados dos itens (b) e (c) podem variar adotando precisão simples ou dupla na declaração das variáveis? justifique apresentando resultados.
 
\end{itemize}

\end{itemize}

\newpage

\text Resolução da Lista utlizando a linguagem	 Python

\subsection*{Python}
\begin{lstlisting}	
import matplotlib.lines as mlines
import matplotlib.pyplot as plt
from math import *

#intervalo de tempo

nParticoes = 4**10
rangeT = range(0, 50)

#coeficientes
K = 0.035871952
theta0 = 99.0
thetaM = 27.0
valorInicial = theta0
deltaT = len(rangeT) / float(nParticoes)

#metodos para resolucao
def solucao_exata(t):
    return (theta0 - thetaM) * exp(K * t * -1) + thetaM

#metodos de aproximação
def metodo_euler_explicito(ini, dt, a):
    return (-1 * a * ini + a * thetaM) * dt + ini

def metodo_euler_implicito(ini, dt, a):
    return ((a * dt * thetaM) + ini) / (1 + a * dt)
    
def metodo_diferenca_central(un, un_ant, dt, a):
    return 2 * a * dt * (thetaM - un) + un_ant

def metodo_crank_nicolson(dt ,a ,un_ant):
    return (un_ant * (1 - ( a * dt)/2) + a * thetaM * dt) / (1 + (a * dt)/2)\end{lstlisting}

\text Gráficos comparando as escolhas de  $\Delta t$:

\graphicspath{{D:/Graficos/}}
\begin{figure}[!htb]
\includegraphics [width=9cm,height=9cm] {4part.png}
\includegraphics [width=9cm,height=9cm]{16part.png}
\end{figure}

\begin{figure}[!htb]
\includegraphics  [width=9cm,height=9cm]{32part.png}
\includegraphics  [width=9cm,height=9cm]{64part.png}
\end{figure}

\begin{figure}[!htb]
\includegraphics  [width=9cm,height=9cm] {256part.png}
\includegraphics  [width=9cm,height=9cm] {1024part.png}
\end{figure}

\end{document}

